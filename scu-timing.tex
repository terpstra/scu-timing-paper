\documentclass{JAC2003}

%%
%%  This file was updated in March 2011 by T. Satogata to be in line with Word templates.
%%
%%  Use \documentclass[boxit]{JAC2003}
%%  to draw a frame with the correct margins on the output.
%%
%%  Use \documentclass[acus]{JAC2003}
%%  for US letter paper layout
%%

\usepackage{graphicx}
\usepackage{booktabs}

%%
%%   VARIABLE HEIGHT FOR THE TITLE BOX (default 35mm)
%%

\setlength{\titleblockheight}{27mm}

\begin{document}
%\title{Facility-Wide Synchronization of Standard FAIR Equipment Controllers}
\title{Timing Characteristics of the FAIR Equipment Controller (SCU)}

\author{
S. Rauch,
W. Terpstra,
W. Panschow,
M. Thieme,
C. Prados,
M. Zweig,
M. Kreider,
D. Beck,
R. B\"ar\\
ANL, Argonne, IL 60439, USA}

\maketitle

\begin{abstract}
The standard equipment controller under development for the new FAIR
accelerator facility is the Scalable Control Unit (SCU). It is designed to
synchronize and control the actions of up to 12 purpose-built slave cards,
connected in a propritary crate by a parallel backplane. Inter-crate
coordination and facility-wide synchronization are a core FAIR requirement
and thus precise timing of SCU slave actions is of vital importance.

The SCU consists primarily of two components, an x86 COM Express daughter
board and a carrier board with an Altera Arria II GX FPGA, interconnected by
PCI Express. The x86 receives configuration and set values with which it
programs the real-time event-condition-action (ECA) unit in the FPGA. The
ECA unit receives event messages via the timing network, which also
synchronizes the clocks of all SCUs in the facility using White Rabbit.
Matching events trigger actions on the SCU slave cards such as: ramping
magnets, triggering kickers, etc.

Timing requirements differ depending on the action taken. For softer
real-time actions, an interrupt can be generated for complex processing on
the x86. Alternatively, the FPGA can directly fire a pulse out a LEMO output
or an immediate SCU bus operation. The delay and synchronization achievable
in each case differs and this paper examines the timing performance of each
to determine which approach is appropriate for the required actions.
\end{abstract}

\section{Introduction}
% blah blah FAIR
% timing offset irrelevant
In the FAIR control system,
a data master issues high-level commands to control accelerator devices.
The front-end controllers in the system react to relevant commands,
issuing appropriate actions to their hardware components.
Depending on the action to be taken,
there are different timing requirements to be met.

Unlike the control system currently deployed at GSI~\cite{old-gsi},
commands issued by the data master carry an absolute execution timestamp.
The front-end controllers must receive commands early enough 
that they can schedule their actions 
to achieve the desired result at the correct time.
Unfortunately, executing actions takes a variable amount of time.
If the action takes 90-110$\mu$s to execute, 
then this places two constraints on the system.
First, the data master must issue commands at least 110$\mu$s ahead of time.
Secondly, the system must be able to tolerate that the action could be as
much as 10$\mu$s too early or too late.

Issuing commands too far in advance reduces the responsiveness of the system.
Once the data master has issued a command, it cannot be aborted.
If the situation changes,
perhaps due to interlock or contention from another beam user,
the system cannot react faster than the slowest action already executing.
This neglects, of course, other sources of latency in the system,
such as network propagation delay, which only exacerbate the problem.
It is thus generally desirable to have fast action execution.

Non-deterministic execution time is a potentially much more serious problem.
For example, if a kicker action executes a few nanoseconds too late,
the beam might be lost.
However, not all actions require the same precision,
and it may make sense to trade accuracy for flexibility in some situations.

Fortunately, the  most common equipment controller in FAIR, 
the Scalable Control Unit (SCU),
has several possible methods for executing actions.
This paper outlines the timing requirements of various accelerator
components in FAIR and explorers the alternatives which could meet them.

\section{Use Cases}
% ask someone who knows timing requirements
% kickers
% ramps
% bunch-to-bunch


\section{Scalable Control Unit (SCU)}
% white rabbit
% figure: block diagram
% what it does
% what it contains
% how it works


\section{Execution Alternatives}
% blah blah -- different options, different trade-offs
When the SCU has an action to perform at a particular time, 
it has many alternative execution paths.
Each option carries a trade-off between timing fidelity and 
the expressiveness of the program which performs the action.

% FPGA
\textbf{FPGA} 
The SCU's FPGA can be programmed to generate the required output
on a phase-aligned 8ns clock edge. 
When augmented by a fine delay card~\cite{cern-fine-delay},
this can be further improved to a general 1ns accuracy.
The only source of non-determinism is the jitter of the FPGA's PLL (ps) 
and the inherent inaccuracy of White Rabbit (ns).
Thus, both the delay and variability are in the sub-nanosecond range 
for this approach.
Unfortunately, this execution path requires custom gateware and/or
a simple output action.

% LM32: no OS
\textbf{LM32} 
Alternatively, the FPGA can issue an interrupt to an embedded soft-CPU (LM32).
This on-chip CPU (with no operating system), 
can then run software to generate the appropriate action.
The delay stems from the time to switch to interrupt context,
run the software routine,
and output the action.
While broadly deterministic,
cache behaviour and on-chip bus accesses contribute to runtime variability.

% Atom: kernel
\textbf{Atom-Kernel}
Venturing futher afield,
the FPGA can issue an interrupt over PCIe to the Atom processor.
The interrupt handler in the kernel driver then takes immediate action.
This delays are the same as for the LM32, 
except that the interrupt is delivered off-chip via the PCIe bus.
Furthermore, 
the Linux kernel may have interrupts masked in some critical sections,
increasing the runtime variability.

% Atom: userspace
\textbf{Atom-Userspace}
Rather than running the execution software in kernel-space,
the SCU could also deliver the interrupt to user-space.
This adds additional context switch overhead, 
but provides for a more comfortable programming environment.

% Atom: FESA
\textbf{FESA}
Finally, the userspace program which executes the action could use the FESA
architecture~\cite{fesa}.
Under this more general framework,
the interrupt is translated to an action using multiple threads.
This again increases the number of context switches and adds inter-process
synchronization delay.
However, it arguably provides the most flexible action execution framework.

\section{Analysis}
% how measured
% graph 1: PDF, free y-scale, 0-100us
% table 2: average, stddev, worst-case spread

% blah blah numbers are good

\section{Conclusion}


%\begin{figure}[htb]
%   \centering
%   \includegraphics*[width=65mm]{JACpic_mc.eps}
%   \caption{Layout of papers.}
%   \label{l2ea4-f1}
%\end{figure}


\begin{thebibliography}{9}   % Use for  1-9  references
%\begin{thebibliography}{99} % Use for 10-99 references

\bibitem{accelconf-ref}
C. Petit-Jean-Genaz and J. Poole, ``JACoW, A service to the Accelerator Community,''
EPAC'04, Lucerne, July 2004, THZCH03,  p.~249, \texttt{http://www.JACoW.org}

\bibitem{jacow-help} A. Name and D. Person, Phys. Rev. Lett. 25 (1997) 56.

\bibitem{exampl-ref}
A.N. Other, ``A Very Interesting Paper,'' EPAC'96, Sitges, June 1996, MOPCH31, p. 7984 (1996),
\texttt{http://www.JACoW.org}  \{no period after URL\}

\bibitem{exampl-ref2}
F.E.~Black et al., {\it This is a Very Interesting Book}, (New York: Knopf, 2007), 52.

\bibitem{exampl-ref3}
G.B.~Smith et al., ``Title of Paper,'' MOXAP07, these proceedings.
\end{thebibliography}

\end{document}
