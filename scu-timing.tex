\documentclass{JAC2003}

%%
%%  This file was updated in March 2011 by T. Satogata to be in line with Word templates.
%%
%%  Use \documentclass[boxit]{JAC2003}
%%  to draw a frame with the correct margins on the output.
%%
%%  Use \documentclass[acus]{JAC2003}
%%  for US letter paper layout
%%

\usepackage{graphicx}
\usepackage{booktabs}

%%
%%   VARIABLE HEIGHT FOR THE TITLE BOX (default 35mm)
%%

\setlength{\titleblockheight}{27mm}

\begin{document}
\title{Facility-Wide Synchronization of Standard FAIR Equipment Controllers}
%\title{Timing Characteristics of the FAIR Equipment Controller (SCU)}

\author{
S. Rauch,
W. Terpstra,
W. Panschow,
M. Thieme,
C. Prados,
M. Zweig,
M. Kreider,
D. Beck,
R. B\"ar\\
ANL, Argonne, IL 60439, USA}

\maketitle

\begin{abstract}
The standard equipment controller under development for the new FAIR
accelerator facility is the Scalable Control Unit (SCU). It is designed to
synchronize and control the actions of up to 12 purpose-built slave cards,
connected in a propritary crate by a parallel backplane. Inter-crate
coordination and facility-wide synchronization are a core FAIR requirement
and thus precise timing of SCU slave actions is of vital importance.

The SCU consists primarily of two components, an x86 COM Express daughter
board and a carrier board with an Altera Arria II GX FPGA, interconnected by
PCI Express. The x86 receives configuration and set values with which it
programs the real-time event-condition-action (ECA) unit in the FPGA. The
ECA unit receives event messages via the timing network, which also
synchronizes the clocks of all SCUs in the facility using White Rabbit.
Matching events trigger actions on the SCU slave cards such as: ramping
magnets, triggering kickers, etc.

Timing requirements differ depending on the action taken. For softer
real-time actions, an interrupt can be generated for complex processing on
the x86. Alternatively, the FPGA can directly fire a pulse out a LEMO output
or an immediate SCU bus operation. The delay and synchronization achievable
in each case differs and this paper examines the timing performance of each
to determine which approach is appropriate for the required actions.
\end{abstract}

\section{Introduction}
% blah blah FAIR
% timing offset irrelevant


\section{Use Cases}
% ask someone who knows timing requirements
% kickers
% ramps
% bunch-to-bunch
The SCU will be mainly used for power supplies for ramped magnets, for controlling
RF devices and for controlling injection kickers.
For controlling the power supplies another device called Adaptive Control Unit (ACU) \cite{acuref} is connected to the SCU. The device controlled in the RF use case is called FPGA Interface Board (FIB) \cite{fibref} . The kicker modules will be controlled by the old MIL-STD-1553 based fieldbus system.


\section{Scalable Control Unit (SCU)}
% figure: block diagram
% what it does
% what it contains
% how it works


\section{Generation Alternatives}
% blah blah -- different options, different trade-offs
% FPGA
% LM32: no OS
% Atom: kernel
% Atom: userspace
% Atom: FESA


\section{Analysis}
% graph 1: PDF, free y-scale, 0-100us
% table 2: average, stddev, worst-case spread

% blah blah numbers are good

\section{Conclusion}


%\begin{figure}[htb]
%   \centering
%   \includegraphics*[width=65mm]{JACpic_mc.eps}
%   \caption{Layout of papers.}
%   \label{l2ea4-f1}
%\end{figure}


\begin{thebibliography}{9}   % Use for  1-9  references
%\begin{thebibliography}{99} % Use for 10-99 references

\bibitem{accelconf-ref}
C. Petit-Jean-Genaz and J. Poole, ``JACoW, A service to the Accelerator Community,''
EPAC'04, Lucerne, July 2004, THZCH03,  p.~249, \texttt{http://www.JACoW.org}

\bibitem{acuref}
H. Ramakers  et al., ``Adaptive Control Unit for Digital Control of Power Converters for Magnets in GSI and FAIR Accelerators,'' GSI Scientific Report 2008, p. 117,
\texttt{http://www-alt.gsi.de/informationen/wti/library/scientificreport2008/PAPERS/GSI-ACCELERATORS-14.pdf}

\bibitem{fibref}
M. Kumm  et al., ``Realtime Communication Based on Optical Fibers for the Control of Digital RF Components,'' GSI Scientific Report 2007, p. 100,
\texttt{http://www-alt.gsi.de/informationen/wti/library/scientificreport2007/PAPERS/GSI-ACCELERATORS-14.pdf}



\bibitem{exampl-ref2}
F.E.~Black et al., {\it This is a Very Interesting Book}, (New York: Knopf, 2007), 52.

\bibitem{exampl-ref3}
G.B.~Smith et al., ``Title of Paper,'' MOXAP07, these proceedings.
\end{thebibliography}

\end{document}
