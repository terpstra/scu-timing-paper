\documentclass{JAC2003}

%%
%%  This file was updated in March 2011 by T. Satogata to be in line with Word templates.
%%
%%  Use \documentclass[boxit]{JAC2003}
%%  to draw a frame with the correct margins on the output.
%%
%%  Use \documentclass[acus]{JAC2003}
%%  for US letter paper layout
%%

\usepackage{graphicx}
\usepackage{booktabs}

%%
%%   VARIABLE HEIGHT FOR THE TITLE BOX (default 35mm)
%%

\setlength{\titleblockheight}{27mm}

\begin{document}
\title{Facility-Wide Synchronization of Standard FAIR Equipment Controllers}
%\title{Timing Characteristics of the FAIR Equipment Controller (SCU)}

\author{
S. Rauch,
W. Terpstra,
W. Panschow,
M. Thieme,
C. Prados,
M. Zweig,
M. Kreider,
D. Beck,
R. B\"ar\\
GSI Helmholtzzentrum f¨ur Schwerionenforschung, D-64291 Darmstadt, Germany}

\maketitle

\begin{abstract}
The standard equipment controller under development for the new FAIR
accelerator facility is the Scalable Control Unit (SCU). It is designed to
synchronize and control the actions of up to 12 purpose-built slave cards,
connected in a proprietary crate by a parallel backplane. Inter-crate
coordination and facility-wide synchronization are a core FAIR requirement
and thus precise timing of SCU slave actions is of vital importance.

The SCU consists primarily of two components, an x86 COM Express daughter
board and a carrier board with an Altera Arria II GX FPGA, interconnected by
PCI Express. The x86 receives configuration and set values with which it
programs the real-time event-condition-action (ECA) unit in the FPGA. The
ECA unit receives event messages via the timing network, which also
synchronizes the clocks of all SCUs in the facility using White Rabbit.
Matching events trigger actions on the SCU slave cards such as: ramping
magnets, triggering kickers, etc.

Timing requirements differ depending on the action taken. For softer
real-time actions, an interrupt can be generated for complex processing on
the x86. Alternatively, the FPGA can directly fire a pulse out a LEMO output
or an immediate SCU bus operation. The delay and synchronization achievable
in each case differs and this paper examines the timing performance of each
to determine which approach is appropriate for the required actions.
\end{abstract}

\section{Introduction}
% blah blah FAIR
% timing offset irrelevant
In the FAIR control system,
a data master issues high-level commands to control accelerator devices.
The front-end controllers in the system react to relevant commands,
issuing appropriate actions to their hardware components.
Depending on the action to be taken,
there are different timing requirements to be met.

Unlike the control system currently deployed at GSI~\cite{old-gsi},
commands issued by the data master carry an absolute execution timestamp.
The front-end controllers must receive commands early enough 
that they can schedule their actions 
to achieve the desired result at the correct time.
Unfortunately, executing actions takes a variable amount of time.
If the action takes 90-110$\mu$s to execute, 
then this places two constraints on the system.
Firstly, the data master must issue commands at least 110$\mu$s ahead of time.
Secondly, the system must be able to tolerate that the action could be as
much as 10$\mu$s too early or too late.

Issuing commands too far in advance reduces the responsiveness of the system.
Once the data master has issued a command, it cannot be aborted.
If the situation changes,
perhaps due to interlock or contention from another beam user,
the system cannot react faster than the slowest action already executing.
This neglects, of course, other sources of latency in the system,
such as network propagation delay, which only exacerbate the problem.
It is thus generally desirable to have fast action execution.

Non-deterministic execution time is a potentially much more serious problem.
For example, if a kicker executes an action a few nanoseconds too late,
the beam might be lost~\cite{kicker}.
However, not all actions require the same precision,
and it may make sense to trade accuracy for flexibility in some situations.

Fortunately, the  most common equipment controller in FAIR, 
the Scalable Control Unit (SCU),
has several possibilities for executing actions.
This paper outlines the timing requirements of various accelerator
components in FAIR and explorers the alternatives which could meet them.

\section{Use Cases}
% ask someone who knows timing requirements
% kickers
% ramps
% bunch-to-bunch
The SCU will be the main frontend controller for the FAIR project.
It provides a uniform platform connected both to the timing- and the data network of the facility. 
In turn, the SCU controls Adaptive Control Units (ACU) \cite{acuref} slaves of various form factors which provide additional
features and the necessary hard- and software interfaces to control the actual accelerator components.
This means a wide range of magnet power supplies, Radio Frequency (RF) components and beam diagnosis equipment.
The set of executable actions of course varies depending on the connected equipment.
A magnet power supply for example will be provided with parameters and timed instructions to source
a current ramp to its magnet, an RF generator gets different sets of frequency and phase parameters and the time when it needs to switch between them. 
The basic concept of the SCU envisions a complete separation of data supply and timing/commands.
This way it can make use of higher abstraction levels ie complex software, which brings flexibility, is more comfortable and maintainable, and also use low level hardware implementations to provide fast, deterministic behavior for the precisely scheduled execution of commands.\\
The device controlled in the RF use case is called FPGA Interface Board (FIB) \cite{fibref}.\\
The kicker modules will be controlled by interface devices called IFK via a MIL-STD-1553
based field bus system.


\section{Scalable Control Unit (SCU)}
% white rabbit
% figure: block diagram
% what it does
% what it contains
% how it works
The SCU is mechaniclly a stack of up to three separated boards. There is the FPGA base board
with an Arria II FPGA, two Small Form-factor Pluggable (SFP) slots, DDR3 ram, parallel flash and a parallel bus (SCU bus) for
controlling up to 12 slave devices. In addition the base boards is equipped with White rabbit circuitry.
A ComExpress module with an Intel Atom CPU is mounted to
the base board. It has ethernet, USB and PCIe interfaces. An optional extension board can
be connected to the base board for backwards compatibility, that runs an MIL-STD-1553 based
field bus interface.\\
The SCU works as a front-end controller. On one side it is connected to the control system via ethernet,
on the other side it controlls slave devices over the SCU bus. It receives timing information over
a White Rabbit link, connected to an SFP. The White Rabbit receiver in the FPGA runs  Precision Time Protocol  (PTP) in software
on a LatticeMico32 (LM32) soft-core CPU. The control system speaks to the Front End Software Architecture (FESA) class running
on the Intel Atom which is connected to the FPGA via a PCIe bridge.
 
\begin{figure}[t]
   \centering
   \includegraphics*[width=\columnwidth]{scu_schema}
   \caption{Block diagram of SCU}
\end{figure}


\section{Execution Alternatives}
% blah blah -- different options, different trade-offs
When the SCU has an action to perform at a particular time, 
it has many alternative execution paths.
Each option carries a trade-off between timing fidelity and 
the expressiveness of the program which performs the action.

% FPGA
\textbf{FPGA} 
The SCU's FPGA can be programmed to generate the required output
on a phase-aligned 8ns clock edge. 
When augmented by a fine delay card~\cite{cern-fine-delay},
this can be further improved to a general 1ns accuracy.
The only source of non-determinism is the jitter of the FPGA's PLL (ps) 
and the inherent inaccuracy of White Rabbit (ns).
Thus, both the delay and variability are in the sub-nanosecond range 
for this approach.
Unfortunately, this execution path requires custom gateware and/or
a simple output action.

% LM32: no OS
\textbf{LM32} 
Alternatively, the FPGA can issue an interrupt to an embedded soft-CPU (LM32).
This on-chip CPU (with no operating system), 
can then run software to generate the appropriate action.
The delay stems from the time to switch to interrupt context,
run the software routine,
and output the action.
While broadly deterministic,
cache behaviour and on-chip bus accesses contribute to runtime variability.

% Atom: kernel
\textbf{Atom-Kernel}
Venturing further afield,
the FPGA can issue an interrupt over PCIe to the Atom processor.
The interrupt handler in the kernel driver then takes immediate action.
This delays are the same as for the LM32, 
except that the interrupt is delivered off-chip via the PCIe bus.
Furthermore, 
the Linux kernel may have interrupts masked in some critical sections,
increasing the runtime variability.

% Atom: userspace
\textbf{Atom-Userspace}
Rather than executing the action's software in kernel-space,
the SCU could also deliver the interrupt to user-space.
This adds additional context switch overhead, 
but provides for a more comfortable programming environment.

% Atom: FESA
\textbf{FESA}
Finally, the userspace program which executes the action could use the FESA
architecture~\cite{fesa}.
Under this more general framework,
the interrupt is translated to an action using multiple threads.
This again increases the number of context switches and adds inter-process
synchronization delay.
However, it arguably provides the most flexible action execution framework.

\section{Analysis}
% how measured
To measure the delay and variability of the alternative execution paths,
we connected two outputs from the SCU to an oscilloscope.
The first output is the action aligned to the FPGA's 8ns clock.
The second output is toggled by the execution path being measured.
This approach excludes the variability of direct FPGA execution.
However, this sub-nanosecond delay is dwarfed by the alternatives measured.

To capture worst-case variability, 
all test systems were subjected to a background work-load
and include at least 10000 samples.
For the LM32, 
we ran the white rabbit PTP core in the background,
and performed a save/restore of all 32 registers on interrupt context switch.
The Atom had a constant background task streaming text over ssh.

We tested the Atom with a real-time patched 2.6.36 Linux kernel.
The PCIe bridge interrupt handler and kernel tasklet processes were set to
real-time priority 99.
For the userspace test, the test program had real-time priority 98.
FESA set its own real-time priority to 60.

\begin{figure}[t]
   \centering
   \begin{tabular}{l|c|c|c|c}
     $\mu$s    & min   & mean  & max   & stddev \\
     \hline
     FPGA      & 0 & 0.001 & 0.001 & 0.001 \\
     LM32      & 2.863 & 2.924 & 3.217 & 0.058  \\
     Kernel    & 7.120 & 13.29 & 37.73 & 3.49   \\
     Userspace & 49.36 & 62.49 & 93.33 & 5.62   \\
     FESA      & 138.9 & 170.1 & 246.1 & 10.8 \\
   \end{tabular}
   \caption{Execution timing performance}
\end{figure}


\begin{figure}[t]
   \centering
   \includegraphics*[width=\columnwidth]{pdf}
   \caption{Comparison of delay distributions}
\end{figure}

We also measured the LM32 without an instruction cache.
This reduced the variability slightly from 354ns to 272ns, 
but greatly increased the average delay from 2.924$\mu$s to 3.810$\mu$s.
Most of the variability appears to stem from pending Wishbone bus operations which
can take multiple cycles to complete.
Our LM32 was clocked at 62.5MHz and when zoomed into the plot around 3$\mu$s,
it becomes obvious that the distribution has 22 spikes with 16ns intervals.
When the background task was removed, 
this degenerates to 2 spikes, 
completely removing the variability.
The 3$\mu$s delay stems mostly from saving and restoring the register set.
In principle, one could eliminate this cost using disjoint registers 
in interrupt and non-interrupt contexts.

From our measurements,
both the time from interrupt to interrupt handler in Linux 
and the time from interrupt handler to userspace completion vary significantly.
Unfortunately, with only 10000 samples, 
we could not trigger the worst case behaviour for both delays simultaneously.
Adding the two worst-case delays measured separately,
we predict that it should be possible for the Userspace delay to hit
120$\mu$s and FESA 280$\mu$s.

\section{Conclusion}
%from Dietrich
With respect for real-time capability,  the timing system for FAIR should meet the performance of the exiisting MIL-1588 based timing system, which are in the order of several microseconds. Unfortunately,
this can not be achieved for code executed in userspace or FESA. Here, times measured not only have a constant offset but a jitter in the same order of magnitude. As a jitter cannot be corrected for in the data master, this requires that all actions requiring harder "real-time" to be executed in the LM32 or in HDL for hard real-time. However, this compromises the control system design that is based for soft-real time actions in FESA for the bulk part of the use cases.


\begin{thebibliography}{9}   % Use for  1-9  references
%\begin{thebibliography}{99} % Use for 10-99 references


\bibitem{acuref}
H. Ramakers  et al., ``Adaptive Control Unit for Digital Control of Power Converters for Magnets in GSI and FAIR Accelerators,'' GSI Scientific Report 2008, p. 117,
\texttt{http://www-alt.gsi.de/informationen/wti/library/scientificreport2008/PAPERS/GSI-ACCELERATORS-14.pdf}

\bibitem{fibref}
M. Kumm  et al., ``Realtime Communication Based on Optical Fibers for the Control of Digital RF Components,'' GSI Scientific Report 2007, p. 100,
\texttt{http://www-alt.gsi.de/informationen/wti/library/scientificreport2007/PAPERS/GSI-ACCELERATORS-14.pdf}



%\bibitem{exampl-ref2}
%F.E.~Black et al., {\it This is a Very Interesting Book}, (New York: Knopf, 2007), 52.

%\bibitem{exampl-ref3}
%G.B.~Smith et al., ``Title of Paper,'' MOXAP07, these proceedings.
\end{thebibliography}

\end{document}
